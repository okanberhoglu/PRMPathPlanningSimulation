\documentclass[12pt]{article}
\usepackage[utf8]{inputenc}
\usepackage{graphicx}
\usepackage{caption}
\usepackage{lipsum}
\usepackage{geometry}
\usepackage{amsmath}
\usepackage{amssymb}
\usepackage[ruled,vlined]{algorithm2e}
\usepackage{float}

\geometry{
    a4paper,        
    left=1.8cm,
    right=1.8cm,
    top=1.5cm,      
    bottom=2.0cm
}

\begin{document}

\title{PRM Path Planning Simulation Project}
\author{Okan Berhoğlu \\ 2739050}
\date{\today}

\maketitle

\section{Abstract}

This project presents the implementation and evaluation of the Probabilistic Roadmap (PRM) path planning algorithm for a PRRR (Prismatic-Revolute-Revolute-Revolute) planar robot arm operating in environments with polygonal obstacles. The PRM algorithm is implemented in this project. The implementation includes a custom configuration distance metric that accounts for the hybrid nature of the configuration space, a randomized inverse kinematics solver for goal configuration generation, and comprehensive collision detection for irregular polygonal obstacles. The system is tested with various goal points and obstacle configurations, demonstrating successful path planning capabilities. Results show that the PRM approach effectively handles the high-dimensional configuration space and produces collision-free paths in cluttered environments.

\section{Introduction}

Path planning is a fundamental problem in robotics, where the goal is to find a collision-free path for a robot to move from an initial configuration to a goal configuration while avoiding obstacles in the environment. The Probabilistic Roadmap (PRM) algorithm is a widely used sampling-based motion planning method that efficiently solves this problem for robots with multiple degrees of freedom operating in complex environments.

This project implements the PRM path planning algorithm for a PRRR (Prismatic-Revolute-Revolute-Revolute) planar robot arm with 4 degrees of freedom. The robot consists of one prismatic joint that allows vertical translation along the x axis and three revolute joints that provide rotational motion.

The PRM algorithm operates in two distinct phases. In the learning phase, a roadmap of collision-free configurations is constructed by randomly sampling the configuration space and connecting nearby configurations with collision-free paths. In the query phase, this precomputed roadmap is used to efficiently solve individual path planning queries by connecting the start and goal configurations to the roadmap and finding the shortest path through the graph structure.

This document is organized as follows. Section 2 describes the implementation method, including the configuration space representation, roadmap construction algorithm, sampling strategy, configuration distance metric, and the query phase with Dijkstra's shortest path algorithm. Section 3 presents the results and visualizations of the implemented PRM planner. Finally, conclusion and references are provided.

\section{Implementation Method}

In this project, Python programming language is used to implement PRM path planning algorithm. And by using matplotlib library, the results are visualized. Besides numpy library is used for 
mathematical operations. The PRM planner is implemented to a PRRR planar robot arm. The robot has 4 degrees of freedom. The configuration space of the robot is defined as $\mathbb{R} \times S^1 \times S^1 \times S^1$.

The configuration space is implemented in the \texttt{Planner} class with the following bounds:
\begin{itemize}
    \item $q_1 \in [-1.0, 1.0]$ meters: The prismatic joint (P) that allows linear translation along the vertical axis
    \item $q_2 \in [-\pi, \pi]$ radians: The first revolute joint (R) rotation angle
    \item $q_3 \in [-\pi, \pi]$ radians: The second revolute joint (R) rotation angle
    \item $q_4 \in [-\pi, \pi]$ radians: The third revolute joint (R) rotation angle
\end{itemize}

The prismatic joint $q_1$ represents a linear space ($\mathbb{R}$), while the three revolute joints $q_2, q_3, q_4$ each represent a circular space ($S^1$) due to their periodic nature. This configuration space structure is crucial for proper distance calculations and interpolation between configurations, as angular differences must account for the wraparound at $\pm\pi$.

PRM divides planning into two phases: the learing phase, during each a roadmap in free space which does not contain any obstacles is built; and the query phase, during which user-defined 
query configurations are connected with the precomputed roadmap. The nodes of the roadmap are configurations in the free space and the edges are collision-free paths between the nodes.
In the implementation phase firstly the roadmap is constructed. Secondly the query phase is implemented. 

\subsection{Roadmap Construction}

The roadmap is represented as a graph where nodes are configurations in the free space and edges are collision-free paths between the nodes. 
In the roadmap construction phase, a number of random configurations are sampled and checked for collisions. If a configuration is collision-free, it is added to the roadmap as a node. 
The given algorithm [1] in the below is used for roadmap construction.

\begin{algorithm}[H]
\caption{Roadmap Construction Algorithm}
\KwIn{$n$: number of nodes to put in the roadmap}
\KwIn{$k$: number of closest neighbors to examine for each configuration}
\KwOut{A roadmap $G = (V, E)$}
$V \leftarrow \emptyset$\;
$E \leftarrow \emptyset$\;
\While{$|V| < n$}{
    \Repeat{$q$ is collision-free}{
        $q \leftarrow$ a random configuration in configuration space;
    }
    $V \leftarrow V \cup \{q\}$\;
}
\For{all $q \in V$}{
    $N_q \leftarrow$ the $k$ closest neighbors of $q$ chosen from $V$ according to $dist$\;
    \For{all $q' \in N_q$}{
        \If{$(q, q') \notin E$}{
            $E \leftarrow E \cup \{(q, q')\}$\;
        }
    }
}
\end{algorithm}

The G is the undirected graph and the V and E are the set of nodes and edges, respectively. Initially, the graph G is empty. Then, for each configuration q in V, the algorithm finds the k closest neighbors of q in V according to the distance function dist. For each neighbor q' of q, if the edge (q, q') is not in E and the path between q and q' is collision-free, then the edge (q, q') is added to E.
The q, q' are the start and end states of the path, respectively. The dist is the distance function between two configurations. The process is repeated until n collision-free configurations have been sampled. 

While calculating the random configurations in the configuration space, uniform sampling method is used.

\begin{algorithm}[H]
\caption{Uniform Random Configuration Sampling}
\KwIn{$q_1^{min}, q_1^{max}$: bounds for prismatic joint}
\KwIn{$q_2^{min}, q_2^{max}, q_3^{min}, q_3^{max}, q_4^{min}, q_4^{max}$: bounds for revolute joints}
\KwOut{A random configuration $q = [q_1, q_2, q_3, q_4]$}
$q_1 \leftarrow$ uniform\_random($q_1^{min}$, $q_1^{max}$)\;
$q_2 \leftarrow$ uniform\_random($q_2^{min}$, $q_2^{max}$)\;
$q_3 \leftarrow$ uniform\_random($q_3^{min}$, $q_3^{max}$)\;
$q_4 \leftarrow$ uniform\_random($q_4^{min}$, $q_4^{max}$)\;
\Return{$q = [q_1, q_2, q_3, q_4]$}\;
\end{algorithm}

In this implementation, the configuration space bounds are:
\begin{itemize}
    \item $q_1 \in [-1.0, 1.0]$ meters (prismatic joint)
    \item $q_2, q_3, q_4 \in [-\pi, \pi]$ radians (revolute joints)
\end{itemize}

Each joint parameter is sampled independently using a uniform distribution over its valid range. This ensures unbiased exploration of the configuration space.

While calculating the closest neighbors of a configuration, the distance function is used. The distance metric must account for the hybrid nature of the configuration space, where $q_1$ is a linear coordinate and $q_2, q_3, q_4$ are angular coordinates on a circle.

For two configurations $q = [q_1, q_2, q_3, q_4]$ and $q' = [q_1', q_2', q_3', q_4']$, the distance is computed as:

\begin{equation}
d(q, q') = \sqrt{d_1^2 + d_2^2 + d_3^2 + d_4^2}
\end{equation}

where:
\begin{align}
d_1 &= |q_1 - q_1'| \\
d_i &= \min(|q_i - q_i'|, 2\pi - |q_i - q_i'|) \quad \text{for } i \in \{2, 3, 4\}
\end{align}

The key difference is that for revolute joints, the angular distance accounts for the wraparound at $\pm\pi$. For example, the distance between $-\pi$ and $\pi$ is 0, not $2\pi$, since they represent the same angular position.

This distance metric ensures that the planner correctly identifies nearby configurations in the configuration space, which is essential for building meaningful edges in the roadmap.

After the distance metric is calculated, the closest neighbors of each nodes are found. To determine the edges of the roadmap, the algorithm checks if the path between the nodes is collision-free. If it is, the edge is added to the roadmap.

\subsection{Query Phase}

In the query phase, the roadmap is used to solve individual path planning problems. Given initial and goal configurations, the algorithm finds a collision-free path between them using the roadmap. 
If successful, the path is returned as the solution. 

The main question is how to connect the initial and goal configurations to the roadmap. The solution involves two steps; first, connecting the start and goal configurations to the roadmap, and second, finding the shortest path through the roadmap using Dijkstra's algorithm.

Both the initial configuration $q_{start}$ and goal configuration $q_{goal}$ are connected to the roadmap by finding their $k$ nearest neighbors and attempting to create collision-free edges. The start configuration is added to the roadmap during initialization, while the goal configuration is added during the query phase.

\begin{algorithm}[H]
\caption{Connect Configuration to Roadmap}
\KwIn{$q$: configuration to connect}
\KwIn{$V$: set of roadmap nodes}
\KwIn{$E$: set of roadmap edges}
\KwIn{$k$: number of neighbors to consider}
\KwOut{Updated roadmap with $q$ connected}
Add $q$ to $V$\;
$neighbors \leftarrow$ empty list\;
\For{each node $v \in V$ (excluding $q$)}{
    $d \leftarrow$ distance($q$, $v$)\;
    Add $(d, v)$ to $neighbors$\;
}
Sort $neighbors$ by distance\;
\For{each of the $k$ closest neighbors $(d, v)$}{
    \If{edge between $q$ and $v$ is collision-free}{
        Add edge $(q, v)$ to $E$\;
    }
}
\end{algorithm}

Once both start and goal configurations are connected to the roadmap, Dijkstra's algorithm is used to find the shortest collision-free path through the roadmap graph.

The algorithm maintains a priority queue ordered by cumulative distance from the start node. It explores nodes in order of increasing distance, guaranteeing that the first path found to the goal is the shortest.

\subsection{Obstacle Generation}

To test the PRM planner in realistic scenarios, random irregular polygonal obstacles are generated in the workspace. The obstacle generation process creates non-overlapping polygons that avoid both the robot's starting configuration and user-specified goal points.

Each obstacle is an irregular polygon with a random number of vertices (typically 3 to 6) positioned around a randomly selected center point $(c_x, c_y)$. The polygon vertices are generated using polar coordinates by first generating $n$ random angles uniformly distributed in $[0, 2\pi]$ and sorting them, then for each angle $\theta_i$, generating a random radius $r_i \in [r_{min}, r_{max}]$, and finally computing the vertex coordinates as $x_i = c_x + r_i \cos(\theta_i)$ and $y_i = c_y + r_i \sin(\theta_i)$.

This approach creates irregular, star-like polygons with varying edge lengths and angles, providing a diverse set of obstacle shapes for testing.

The obstacle generation process includes several collision checks to ensure valid obstacle placement. Obstacles must not collide with the robot's initial configuration, must not contain user-specified goal points, and must not overlap with previously generated obstacles. Collision detection is performed using three geometric algorithms: segment intersection checks if two line segments intersect using cross products, point-in-polygon test uses ray casting algorithm to determine if a point lies inside a polygon, and polygon-polygon collision combines edge intersection tests and point containment checks. The generation process uses rejection sampling with a maximum number of attempts to ensure termination even in crowded workspaces. If a randomly generated polygon fails any collision check, it is discarded and a new polygon is generated.

\subsection{Inverse Kinematics for Goal Configuration}

To connect the goal point in workspace to a configuration in the configuration space, an inverse kinematics (IK) solution is required. Since analytical IK solutions for the PRRR robot can be complex, a randomized sampling-based approach is used. The algorithm randomly samples configurations from the configuration space, computes the forward kinematics to determine the end-effector position, and selects the collision-free configuration that minimizes the distance to the target goal point. The process continues for a maximum number of attempts or until a configuration is found within a specified tolerance.

\section{Results}

After the PRM algorithm is implemented, it is tested with different parameters. The results are given in this section.

First, the algorithm is tested with an easily reachable goal point for the robot. The figures given as figure 1 and figure 2 shows the result of this test. The first figure shows the workspace with the robot and the goal point. The second figure shows the roadmap generated by the algorithm in the configuration space. 

\begin{figure}[H]
    \centering
    \includegraphics[width=0.5\textwidth]{result_1.png}
    \caption{The workspace with the robot and the goal point.}
    \label{fig:result_1}
\end{figure}

\begin{figure}[H]
    \centering
    \includegraphics[width=0.98\textwidth]{result_1_c.png}
    \caption{The roadmap generated by the algorithm in the configuration space of the results shown in figure 1.}
    \label{fig:result_1_c}
\end{figure}

Since the configuration space is 4 dimensional, the visualization of the roadmap is seperated as q1, q2 and q3, q4 links visualization. 

\begin{figure}[H]
    \centering
    \includegraphics[width=0.5\textwidth]{result_2.png}
    \caption{The workspace with the robot and the goal point.}
    \label{fig:result_2}
\end{figure}

\begin{figure}[H]
    \centering
    \includegraphics[width=0.98\textwidth]{result_2_c.png}
    \caption{The roadmap generated by the algorithm in the configuration space of the results shown in figure 3.}
    \label{fig:result_2_c}
\end{figure}

Figure 4 displays the roadmap generated in the configuration space corresponding to the workspace results shown in Figure 3. As demonstrated in Figure 3, the robot successfully reaches the goal configuration within the specified tolerance.

\begin{figure}[H]
    \centering
    \includegraphics[width=0.5\textwidth]{result_3.png}
    \caption{The workspace with the robot and the goal point.}
    \label{fig:result_3}
\end{figure}

\begin{figure}[H]
    \centering
    \includegraphics[width=0.98\textwidth]{result_3_c.png}
    \caption{The roadmap generated by the algorithm in the configuration space of the results shown in figure 5.}
    \label{fig:result_3_c}
\end{figure}

There is a hard end configuration for the robot given in the figure 5. The given goal point is close to robots initial state and there is an obstacle between the robot and the goal point. These conditions makes hard for the robot to reach the goal position.
However, the algorithm works well and finds a path for the robot to reach the goal position within the specified tolerance in mentioned conditions.

\begin{figure}[H]
    \centering
    \includegraphics[width=0.5\textwidth]{result_4.png}
    \caption{The workspace with the robot and the goal point.}
    \label{fig:result_4}
\end{figure}

\begin{figure}[H]
    \centering
    \includegraphics[width=0.98\textwidth]{result_4_c.png}
    \caption{The roadmap generated by the algorithm in the configuration space of the results shown in figure 7.}
    \label{fig:result_4_c}
\end{figure}

Figure 7 demonstrates a scenario where the robot cannot reach the specified goal point because it lies outside the robot's reachable workspace. However, the planner successfully generates a path to the reachable configuration that is closest to the desired goal.

\begin{figure}[H]
    \centering
    \includegraphics[width=0.5\textwidth]{result_5.png}
    \caption{The workspace with the robot and the goal point.}
    \label{fig:result_5}
\end{figure}

In Figure 9, the obstacle prevents the robot from reaching the goal point directly. However, the algorithm successfully finds a path to the closest reachable configuration.

\begin{figure}[H]
    \centering
    \includegraphics[width=0.5\textwidth]{result_6.png}
    \caption{The workspace with the robot and the goal point.}
    \label{fig:result_6}
\end{figure}

In Figure 10, the robot cannot reach the goal because it is located far beyond the robot's initial reach. Nevertheless, the algorithm successfully plans a path to the closest reachable point.

\begin{figure}[H]
    \centering
    \includegraphics[width=0.5\textwidth]{result_7.png}
    \caption{The workspace with the robot and the goal point.}
    \label{fig:result_7}
\end{figure}

\begin{figure}[H]
    \centering
    \includegraphics[width=0.98\textwidth]{result_7_c.png}
    \caption{The roadmap generated by the algorithm in the configuration space of the results shown in figure 11.}
    \label{fig:result_7_c}
\end{figure}

Figure 11 demonstrates a scenario where the robot successfully reaches the goal within the specified tolerance. Figure 12 shows the corresponding configuration space roadmap generated by the algorithm.
\section{Conclusion}

This project successfully implemented a Probabilistic Roadmap (PRM) planner for a 4-DOF PRRR planar robot. The system effectively handles a hybrid configuration space (one prismatic, three revolute joints) moving among polygonal obstacles. Key contributions include a custom distance metric that correctly handles angular periodicity and a randomized inverse kinematics solver for goal generation.
    
The planner builds a connectivity graph using uniform sampling and k-nearest neighbors, then uses Dijkstra's algorithm to find collision-free paths. Tests in various cluttered environments showed the system is robust and capable of finding valid paths even in challenging scenarios.
    
Future improvements could include adaptive sampling for better coverage in narrow passages, path smoothing for more natural motion, and lazy collision checking to speed up query times.

\begin{thebibliography}{9}

\bibitem{choset2005}
Choset, H., Lynch, K. M., Hutchinson, S., Kantor, G. A., \& Burgard, W. (2005).
\textit{Principles of Robot Motion: Theory, Algorithms, and Implementations.}
MIT Press.

\end{thebibliography}

\end{document}